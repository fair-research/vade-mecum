\documentclass[11pt]{article}

\input preamble

\newdimen\tlx
\newdimen\tlx
\newdimen\brx
\newdimen\bry

\newif\iffinal

% Un-comment this line to see proposal without comments
%\finaltrue

\iffinal
    \newcommand\ian[1]{}
\else
    \newcommand\ian[1]{{\color{red}[Ian: #1]}}
 \fi

\newif\iffancy

% Un-comment this line to see proposal for distribution
\fancytrue

\begin{document}



The related concepts of FAIR and reproducible research.


\section{Why FAIR?}

It makes you more productive, because first of all you are making results findable, accessible, etc. by the future you.

It makes your lab more productive ... FAIR by YOU (and your team).

It makes your research more visible.

It's fun, once you get the hang of it...

\ian{We could include here some counter-arguments to why NOT fair}

You might be mocked.

You might be scooped.

It's too time-consuming.

\section{10 simple rules}

10 simple rules for reproducible computational research defined by Sandve et al.~\cite{sandve2013ten}:

Basically all 

\subsection{For every result, keep track of how it was produced} 
We preserve workflows and assign Minids to workflow results.

\subsection{Avoid manual data manipulation steps} 
We encode all data manipulation steps in either Galaxy workflows or R scripts.

\subsection{Archive the exact versions of all external programs used} 
We create a Docker container with versions of the tools used in the analysis, 
and generate Minids for the Docker file and Docker image of the container.

\subsection{Version control all custom scripts} 
We maintain our programs in GitHub, which supports versioning,
and provide Minids for the versions used.

\subsection{Record all intermediate results, when possible in standardized formats}

\subsection{For analyses that include randomness, note underlying random seeds} 



\subsection{Always store raw data behind plots} 



\subsection{Generate hierarchical analysis output, allowing layers of increasing detail to be inspected} 

\subsection{Connect textual statements to underlying results} 

Your paper includes the startling conclusion that ``the answer to the ultimate question is 42." 
Of course, if you rely on some authority in making that statement, you would cite that authority~\cite{Adams}.
Similarly, if that result is obtained via some computational analysis, you should describe how it was obtained.

\ian{We'd add: And any number that appears in a paper!}

\subsection{Provide public access to scripts, runs, and results} 

\end{document}


