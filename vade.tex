\documentclass[11pt]{article}

\input preamble

\newdimen\tlx
\newdimen\tlx
\newdimen\brx
\newdimen\bry

\newif\iffinal

% Un-comment this line to see proposal without comments
%\finaltrue

\iffinal
    \newcommand\ian[1]{}
    \newcommand\carl[1]{}
\else
    \newcommand\ian[1]{{\color{red}[Ian: #1]}}
    \newcommand\carl[1]{{\color{blue}[Carl: #1]}}
 \fi

\newif\iffancy

% Un-comment this line to see proposal for distribution
%\fancytrue


\begin{document}
\title{FAIR Research: A Vade Mecum}
\author{}
\date{}
\maketitle

\section{Introduction}

Do you want your research to contribute to knowledge?
Do you want your work to be useful to others?
Do you want your work today to be useful to ``future you": yourself a year from now?
If your answer to any of these questions is ``yes," 
then you should want to organize your work so that its products are
\textbf{F}indable, \textbf{A}ccessible, \textbf{I}nteroperable, and \textbf{R}eusable: 
that is, so that they are \textbf{FAIR}.
This document is intended as a handy guide---a \emph{vade mecum}---to  
methods that you can use to achieve these goals.

\ian{A few lines of philosophy: simple, straightforward, informal.... no best way for all people and 
all circumstances, but simple methods that can be make a difference. (Need to circumscribe to computational somehow?)}

% See http://qresp.org

\section{Why FAIR?}

Others have spent much time and ink explaining the importance of the FAIR principle~\cite{wilkinson16},
citing, for example, the purported reproducibility crisis in research and the need to democratize access to the
results of research. 
Those motivations are important, but if you're like us, 
you probably care care even more about getting your paper or proposal out before its deadline tumbles under the weight of those that follow. 
Thus, we'll focus here on more self-serving reasons for FAIR:

% Point to the moral tone of the word FAIR :)

\begin{itemize}
\item
\emph{It makes you more productive.}
%We find it helpful to bear in mind that the first consumer for FAIR research is ourselves.
 because first of all you are making results useful by the future you.

\item
\emph{It makes your lab more productive.} 

\item
\emph{It makes your research more visible.}

\item
\emph{It's fun, once you get the hang of it.} 
Perhaps this is just us, but we find that ...
\end{itemize}

We'd also like to address some excuses that we hear for ... 

\begin{itemize}
\item
``I'll be scooped."

\item
``I don't have time."

\item
``I'll be mocked."
 
\end{itemize}


\section{Ten simple rules}

Geir Sandve and colleagues defined 10 simple rules for reproducible computational research~\cite{sandve2013ten}.

\subsection{For every result, keep track of how it was produced} 


We preserve workflows and assign Minids to workflow results.

\subsection{Avoid manual data manipulation steps} 

Surely the biggest obstacle to FAIR research must be the quick scaling performed on a calculator, 
the manual edit to a data file's format, 
or the cut and paste from a table. 
You record the results, but a week later you are hard pressed to recall what exactly was done.
There's a gap in your logic, and if you made a mistake, that mistake is invisible.
Like a pothole in a highway, its enough to run your well-oiled machine off the road.

Write a script.
We encode all data manipulation steps in either Galaxy workflows or R scripts.
Sometimes, a shell script is sufficient.



\subsection{Archive the exact versions of all external programs used} 
We create a Docker container with versions of the tools used in the analysis, 
and generate Minids for the Docker file and Docker image of the container.

\subsection{Version control all custom scripts} 
We maintain our programs in GitHub, which supports versioning,
and provide Minids for the versions used.

\subsection{Record all intermediate results, when possible in standardized formats}

\subsection{For analyses that include randomness, note underlying random seeds} 



\subsection{Always store raw data behind plots} 

%When you launched on a career in research, 
%did you ever imagine that you would find yourself reverse engineering 


\subsection{Generate hierarchical analysis output, allowing layers of increasing detail to be inspected} 

\subsection{Connect textual statements to underlying results} 

Your paper includes the startling conclusion that ``the answer to the ultimate question is 42." 
Of course, if you relied on some authority in making that statement, you would cite that authority~\cite{Hitch}.
Similarly, if that result is obtained via some computational analysis, you should describe how it was obtained.
Perhaps, for example, the number 42 is the mean of XXX. 


\subsection{Provide public access to scripts, runs, and results}


\bibliographystyle{plain}
\bibliography{refs}

\end{document}


